\section*{Tema 5}
\addcontentsline{toc}{section}{Tema 5}
%    \subsection*{Oráculo}
    \addcontentsline{toc}{subsection}{Oráculo}
%    \subsection*{Pruebas unitarias}
    \addcontentsline{toc}{subsection}{Pruebas unitarias}
%    \subsection*{Pruebas de integración}
    \addcontentsline{toc}{subsection}{Pruebas de integración}
%    \subsection*{Pruebas de sistema}
    \addcontentsline{toc}{subsection}{Pruebas de sistema}
%    \subsection*{Pruebas de aceptación}
    \addcontentsline{toc}{subsection}{Pruebas de aceptación}
%    \subsection*{Breaking software}
    \addcontentsline{toc}{subsection}{Breaking softwar}
%    \subsection*{Planes de prueba}
    \addcontentsline{toc}{subsection}{Planes de prueba}
%    \subsection*{Caso de prueba}
    \addcontentsline{toc}{subsection}{Caso de prueba}
    \subsection*{Ciclo de vida del defecto}
    \addcontentsline{toc}{subsection}{Ciclo de vida del defecto}
        Al momento de informar un defecto, su vida apenas comienza. Este ciclo consiste en 5 etapas:
        
        \begin{enumerate}
            \item Descubrimiento: el defecto es encontrado
            \item Reporte: el defecto es registrado, se documenta evidencia
            \item Asignación: el defecto es entregado para ser atendido
            \item Reparación: el código afectado se refactoriza
            \item Verificación: se comprueba la funcionalidad o comportamiento
        \end{enumerate}

        